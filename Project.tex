\documentclass[11pt]{article}
\usepackage{amsmath}
\usepackage{amssymb}
\usepackage{fancyhdr}
\usepackage{graphicx}
\usepackage{pdfpages}
\usepackage{listings}
\usepackage{color}
\usepackage{color}
\definecolor{dkgreen}{rgb}{0,0.6,0}
\definecolor{gray}{rgb}{0.5,0.5,0.5}
\definecolor{mauve}{rgb}{0.58,0,0.82}
\pagestyle{fancy}
\setlength{\oddsidemargin}{0in}
\setlength{\evensidemargin}{0in}
\setlength{\textheight}{9in}
\setlength{\textwidth}{6.5in}
\setlength{\topmargin}{-0.5in}
\setlength{\headheight}{14pt}

\lstset{language=R,
	aboveskip=3mm,
	belowskip=3mm,
	showstringspaces=false,
	basicstyle={\small\ttfamily},
	numbers=none,
	numberstyle=\tiny\color{gray},
	keywordstyle=\color{blue},
	commentstyle=\color{dkgreen},
	stringstyle=\color{mauve},
	breaklines=true,
	breakatwhitespace=true,
	tabsize=4
}


%%%%%%%%%%%%%%%%%%%%%%%%%%%%%
\title{\vspace{-3ex}\bf Final Project\\[2ex] 
       \normalsize ECS 132 --- Winter 2015}
\date{\today}
\author{\bf William Otwell (\#\#\#\#\#\#\#\#\#)\\ \bf Rupali Saiya (\#\#\#\#\#\#\#\#\#)\\ \bf Nicholas Layton(996933702)\\ \bf Syeda Inamdar(\#\#\#\#\#\#\#\#\#)\\}

\begin{document}
\maketitle
\pagebreak
\tableofcontents
\pagebreak

%%%%%%%%%%%%%%%%%%%%%%%%%%%%%
\section{Problem 1}
\label{sec:problem1}
\subsection{Part A: Comparison of Two Means}
\label{subsec:1a}
The bike sharing dataset provided by the UC Irvine (UCI) Machine Learning Repository allows us to analyze bike rental statistics during the years 2011 and 2012. We decided to analyze and compare the quantity of bike rentals on days during the months of March 2011 and March 2012 to gain insight as to how much the bike rental rates changed from one year to the next. In more formal terms, we estimated the difference between the average number of bikes rented in March 2011 and the average number of bikes rented in March 2012. To do so, we constructed a confidence interval from the sample provided by UCI to estimate the difference between the population means of the averages from each month. We first had to define our random variables before constructing our confidence interval. Below is a list of the random variables that we used:
\begin{itemize}
	\item $X$: the number of bikes rented on any given day in March 2011
	\item $Y$: the number of bikes rented on any given day in March 2012
	\item $\bar{X}$: the sample mean of the number of bikes rented per day in March 2011
	\item $\bar{Y}$: the sample mean of the number of bikes rented per day in March 2012
	\item $s_1$: the standard deviation of the number of bikes rented per day in March 2011
	\item $s_2$: the standard deviation of the number of bikes rented per day in March 2012
\end{itemize}
An important aspect worth noting is that the random variables $X$ and $Y$ are independent of one another. After defining our random variables, we had to construct our model for constructing the confidence interval. Our model turned out to be the following:
\begin{equation}
\bar{X} - \bar{Y} \pm (1.96) \sqrt{\frac{s_{1}^{2}}{n_1} + \frac{s_{2}^{2}}{n_2}}
\end{equation}
\dots where $n_1$ and $n_2$ were the number of days in March 2011 and March 2012 (respecfully) for which data was recorded. The R code that we developed to calculate this confidence interval can be found in Appendix~\ref{sec:problem1code}.

After running our R calculations, we found that the average number of bike rentals in March 2011 was 2065.968 bikes and that of March 2012 was 5318.548 bikes. The difference between the means of our two samples was -3252.581. The standard deviation of the number of bikes rented in March 2011 was \#\#\#\#, and the standard deviation of that in March 2012 was \#\#\#\#. From these values, we found our confidence interval to be $(-5932.108, -573.0535)$. In formal terms, we are 95\% confident that the difference between the mean number of bike rentals on days in March 2011 and days in March 2012 lies within the interval ranging from $-5932.108$ to $-573.0535$.  

Although this is a very wide ranging interval, we still were able to make inferences about the bike rental trends in March 2011 versus the trends in March 2012. Being that we subtracted the sample mean of the bike rentals in March 2012 from that of the bike rentals in March 2011, we inferred that there were significantly more bike rentals in March 2012 than there were in March 2011. This is because the entirety of the interval lies below 0. In the next section, we used a proportion of warmer days in the same two months to see if temperature could have influenced the increase in the number of bike rentals from March 2011 to March 2012.

\subsection{Part B: Comparison of Two Proportions}
\label{subsec:1b}
We wanted to compare the proportion of warmer than average days in March 2011 to those in March 2012. We constructed a confidence interval to estimate the difference between population proportions of the number of warmer days in March 2011 and March 2012. Again, we defined our random variables and then constructed the confidence interval. 
\begin{itemize}
	\item $\hat{p_1}$: the sample proportion of warmer days in March 2011
	\item $\hat{p_2}$: the sample proportion of warmer days in March 2012
	\item $s_{1}^{2}$: the standard deviation of the proportion of warmer days in March 2011
		\subitem Mathematical expression: $s_{1}^{2} =\hat{p_1}(1 - \hat{p_1})$
	\item $s_{2}^{2}$: the standard deviation of the proportion of warmer days in March 2012
		\subitem Mathematical expression: $s_{2}^{2} = \hat{p_2}(1 - \hat{p_2})$
\end{itemize}
The random variables $\hat{p_1}$ and $\hat{p_2}$ are independent of one another. The model that we used to calculate the confidence interval for the difference in the two proportions was:
\begin{equation}
\hat{p_1} - \hat{p_2} \pm (1.96) \sqrt{\frac{s_1^2}{n_1}+\frac{s_2^2}{n_2}} 
\end{equation}
\begin{equation}
= \hat{p_1} - \hat{p_2} \pm (1.96) \sqrt{\frac{\hat{p_1}(1 - \hat{p_1})}{n_1}+\frac{\hat{p_1}(1 - \hat{p_1})}{n_2}}
\end{equation}
\dots where $n_1$ and $n_2$ are the number of days in March 2011 and March 2012 (respectfull) during which the temperature was warm. We defined a "warmer day" to be one in which the temperature rose above 12.5 degrees Celsius, which translated to be 0.3 in the UCI dataset. The R code that was developed to evaluate the above expression can be found in Appendix~\ref{sec:problem1code}.

In our calculation we found that the proportion of warmer days in March 2011 was 0.5806452 and that the proportion of warmer days in March 2012 was 0.9032258. The difference between the proportions of our two samples was found to be 0.3225833. The standard deviation of the proportion of warmer days in March 2011 was \#\#\#\#, and the standard deviation of the proportion of warmer days in March 2012 was \#\#\#\#. The resulting confidence interval was $(-0.5250816, -0.1200797)$. We can say that we are 95\% confident that the difference between the population proportions for warmer days in the months of March 2011 and March 2012 lies within the interval ranging from $-0.5250816$ to  $-0.1200797$.

These two confidence intervals show that there could be a relationship between the proportion of warmer days and the number of bike rentals in the month of March. We can see that both the proportion of warmer days and the number of bike rentals in March 2012 were greater than that of March 2011, which suggests that warmer weather could have influenced more people to rent bikes. Although it makes intuitive sense that there is a relationship between warmer weather and an increase in bike rentals, our calculations cannot ensure that warmer weather was, in fact, the reason why there were more people renting bikes. We can only say that there is a potential relationship, given that the confidence intervals indicate as such. 
\pagebreak



%%%%%%%%%%%%%%%%%%%%%%%%%%%%%
\section{Problem 2}
\label{sec:problem2}
\subsection{Part A}
\label{subsec:2a}
\subsection{Part B}
\label{subsec:2b}
\subsection{Part C}
\label{subsec:2c}
\subsection{Part D}
\label{subsec:2d}
\subsection{Part E}
\label{subsec:2e}
\subsection{Part F}
\label{subsec:2f}
\subsection{Part G}
\label{subsec:2g}
\subsection{Part H}
\label{subsec:2h}
\pagebreak

%%%%%%%%%%%%%%%%%%%%%%%%%%%%%
\section{Problem 3}
\label{sec:problem3}
\subsection{Part A: Nonparametric Density Estimation}
\label{subsec:3a}


\subsection{Part B: Fitting a Parametric Model}
\label{subsec:3b}
In the previous section, we formed a nonparametric estimate of the density of the daily temperature data by constructing a histogram. After analyzing the histogram, we chose to model its density with a normal parametric distribution. The histogram showed two peaks: one peak among the values that represented colder weather and another among those of warmer temperatures. It also showed a dip among values that are somewhere between colder and warmer temperatures. These observations suggested that the daily temperature from 2011 to 2012 in Washington D.C. matched that of a bimodal distribution, however, we theorized that the population temperature, in fact, could have been normally distributed.  

We thought that the frequency of days in which temperature was either extremely cold or extremely warm should be less than that of days in which the temperature was somewhere in between. Our theory came from the idea that climates seem to rarely shift from hot to cold (or vice-versa) in short periods of time, meaning there should be an extended period of time between in which the weather is of moderate temperature. For this reason, we thought there should be a larger number of days of moderate temperature, and that the data should have represented a more normal distribution. 

In the following sections we explain how we fitted our data to a normal parametric model using two different approaches: the Method of Moments approach, and the Maximum Likelihood approach. 
\subsubsection{Method of Moments}
\label{subsubsec:methodofmoments}
The method of moments involves equation population parameters for a particular parametric distribution to its population moments. In the case of the normal distribution these parameters are the mean and the variance. To accurately fit the normal parametric model to our data, we had to determine the method of moments estimators for the mean and variance. We found the first and second theoretical moments to be: 
\begin{equation}
E(X_i) = \mu\ \ \&\ \ E(X_i^2) = \sigma ^ 2 + \mu ^ 2
\end{equation}
The second equation came from manipulating the formula for variance. We cam up with two equations to estimate the moments based on our sample data:
\begin{equation}
E(X_i) = \mu = \frac{1}{n} \sum_{i=1}^n X_i
\end{equation}
\begin{equation}
E(X_i^2) = \sigma ^ 2 + \mu ^ 2 = \frac{1}{n} \sum_{i=1}^{n} X_i^2
\end{equation}
We were able to calculate these values from our sample data of daily temperatures using R. The code that we developed to calculate these moments can be found in Appendix~\ref{sec:problem3code} in the function labeled "PartB()".
\subsubsection{Maximum Likelihood}
\label{subsubsec:maximumlikelihood}

\subsection{Part C: Plotting Parametric-Model Curves}
\label{subsec:3c}
\pagebreak

%%%%%%%%%%%%%%%%%%%%%%%%%%%%%
\appendix
\section{Problem 1 Code}
\label{sec:problem1code}
\lstinputlisting{Problem1.R}
\pagebreak

%%%%%%%%%%%%%%%%%%%%%%%%%%%%%
\section{Problem 2 Code}
\label{sec:problem2code}
\lstinputlisting{Problem2.R}
\pagebreak

%%%%%%%%%%%%%%%%%%%%%%%%%%%%%
\section{Problem 3 Code}
\label{sec:problem3code}
\lstinputlisting{Problem3.R}
\pagebreak

%%%%%%%%%%%%%%%%%%%%%%%%%%%%%
\section{Problem 3 Plots}
\label{sec:problem3plots}
\subsection{Part A}
\label{sec:problem3aplot}
\includegraphics{Problem3A.pdf}

\subsection{Part C: Method of Moments}
\label{subsesc:problem3moments}
\includegraphics{Problem3CMoments.pdf}
\newline
\pagebreak

\subsection{Part C: Maximum Likelihood}
\label{subsesc:problem3cmaximumlikelihood}
\includegraphics{Problem3CMLE.pdf}
\newline
\pagebreak

\subsection{Part C: Combined}
\label{subsesc:problem3ccombined}
\includegraphics{Problem3C.pdf}
\newline
\pagebreak

%%%%%%%%%%%%%%%%%%%%%%%%%%%%%
\section{Group Member Contributions}
\begin{itemize}
	\item William Otwell:
	\begin{enumerate}
		\item Planned and conceptualized the solution for Problem 1
		\item Helped develop and debug R code for Problem 1
	\end{enumerate}
	
	\item Rupali Saiya:
	\begin{enumerate}
		\item Planned and conceptualized the solution for Problem 1
		\item Helped develop the Problem 1 Write-Up
		\item Planned and conceptualized the approach to solving Problem 3: Part B
	\end{enumerate}
	
	\item Nicholas Layton:
	\begin{enumerate}
		\item Developed R code for Problem 1: Part A
		\item Developed the R code for Problem 3: Part A
		\item Planned and conceptualized the approach to solving Problem 3: Part B
	\end{enumerate}
	
	\item Syeda Inamdar:
	\begin{enumerate}
		\item Developed R code for Problem 1: Part B
		\item Planned and conceptualized the approach to solving Problem 3: Part B
	\end{enumerate}
\end{itemize}
\pagebreak

\end{document}

