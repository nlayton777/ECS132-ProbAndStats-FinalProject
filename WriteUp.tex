\documentclass[11pt]{article}
\usepackage{amsmath}
\usepackage{amssymb}
\usepackage{fancyhdr}
\usepackage{graphicx}
\usepackage{listings}
\usepackage{color}
\usepackage{color}
\definecolor{dkgreen}{rgb}{0,0.6,0}
\definecolor{gray}{rgb}{0.5,0.5,0.5}
\definecolor{mauve}{rgb}{0.58,0,0.82}
\pagestyle{fancy}
\setlength{\oddsidemargin}{0in}
\setlength{\evensidemargin}{0in}
\setlength{\textheight}{9in}
\setlength{\textwidth}{6.5in}
\setlength{\topmargin}{-0.5in}
\setlength{\headheight}{14pt}

\lstset{language=R,
	aboveskip=3mm,
	belowskip=3mm,
	showstringspaces=false,
	basicstyle={\small\ttfamily},
	numbers=none,
	numberstyle=\tiny\color{gray},
	keywordstyle=\color{blue},
	commentstyle=\color{dkgreen},
	stringstyle=\color{mauve},
	breaklines=true,
	breakatwhitespace=true,
	tabsize=4
}


%%%%%%%%%%%%%%%%%%%%%%%%%%%%%
\title{\vspace{-3ex}\bf Final Project\\[2ex] 
       \normalsize ECS 132 --- Winter 2015}
\date{\today}
\author{\bf William Otwell (\#\#\#\#\#\#\#\#\#)\\ \bf Rupali Saiya (\#\#\#\#\#\#\#\#\#)\\ \bf Nicholas Layton(996933702)\\ \bf Syeda Inamdar(\#\#\#\#\#\#\#\#\#)\\}

\begin{document}
\maketitle
\pagebreak
\tableofcontents
\pagebreak

%%%%%%%%%%%%%%%%%%%%%%%%%%%%%
\section{Problem 1}

In this problem we decided to compare the number of bikes renting in March 2011 versus March 2012. We found that the mean number of bike rentals in March 2011 was 2065.968 bikes and in March 2012 was 5318.548 bikes. The difference between the means of our two samples was found to be -3252.581. We are  95\% sure that the interval from -5932.108 to -573.0535 contains the mean population difference. 
\\

This interval is very large, but shows a significant increase in bike ridership from March 2011 to March 2012. We will use a proportion of warmers days in March to see if this could have affected the increase in the number of bike rentals in March 2012.
\\

We wanted to find the proportion of warmer than average days in March 2011 versus March 2012. To do this we calculated the proportion of days where the temperature was over 12.5 degrees Celsius (a value greater than 0.3 from the data we were given) in March 2011 versus March 2012. We found that the proportion of days warmer than 12.5 degrees Celsius in March 2011 was 0.5806452 and in March 2012 was 0.9032258. The difference between the proportions of our two samples was found to be 0.3225833 and the 95\% confidence interval for this data was -0.5250816 to -0.1200797. 
\\

These proportions show that there is a correlation between warmer days and bike rentals in the month of March. We can see that the proportion of warmer days in March 2012 was greater than that of March 2011, which correlates to the increase in bike ridership in March 2012 as well.

\pagebreak



%%%%%%%%%%%%%%%%%%%%%%%%%%%%%
\section{Problem 2}
\pagebreak

%%%%%%%%%%%%%%%%%%%%%%%%%%%%%
\section{Problem 3}
\pagebreak

%%%%%%%%%%%%%%%%%%%%%%%%%%%%%
\appendix
\section{Problem 1 Code}
\lstinputlisting{Problem1.R}
\pagebreak

%%%%%%%%%%%%%%%%%%%%%%%%%%%%%
\section{Problem 2 Code}
\lstinputlisting{Problem2.R}
\pagebreak

%%%%%%%%%%%%%%%%%%%%%%%%%%%%%
\section{Problem 3 Code}
\lstinputlisting{Problem3.R}
\pagebreak

\end{document}

